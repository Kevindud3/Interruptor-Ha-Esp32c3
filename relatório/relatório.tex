% Topic: relatório
% Created: 21-05-2025

\documentclass[]{abntex2}
\usepackage[T1]{fontenc}
\usepackage[]{graphicx}
\usepackage[]{circuitikz}
\usepackage[]{hyperref}
\usepackage[]{amsmath}
\usepackage[portuguese]{babel}
\usepackage{booktabs}
\newtheorem{exemplo}{Exemplo}
\newcommand{\diff}{\ensuremath{\operatorname{d}\!}}
\usepackage[portuguese]{babel}
\usepackage{minted}

\begin{document}

\chapter{Desenvolvimento}
\section{Lâmpada}
Este projeto tem como objetivo o desenvolvimento de uma lâmpada inteligente com
controle de intensidade luminosa (\textit{dimmer}) e acionamento automático
baseado em sensores de luminosidade e presença. Para isso, foram utilizados um
microcontrolador, o sensor de luminosidade BH1750 e o sensor de presença por
radar LD2410C, além da integração com o sistema de automação residencial Home
Assistant OS (HAOS) via protocolo MQTT.

O desenvolvimento iniciou-se com uma prova de conceito, validando a comunicação
MQTT entre o microcontrolador e o \textit{addon} \textit{MQTT Broker} no HAOS.
A partir dessa etapa, foram realizadas as integrações dos sensores e a criação
de automações no Home Assistant.

\subsection{Conexão Wi-Fi}

A conexão Wi-Fi foi implementada com base no exemplo oficial disponível na
biblioteca Arduino
WiFi.\footnote{\url{https://github.com/arduino-libraries/WiFi/blob/master/examples/ConnectWithWPA/ConnectWithWPA.ino}}

O SSID e a senha da rede são armazenados em variáveis, e a função
\texttt{WiFi.begin()} é utilizada para iniciar a conexão.

O status da conexão é exibido no monitor serial, permitindo a verificação do
sucesso da conexão e facilitando a depuração.

\begin{minted}{cpp}
const char* ssid = "HA AP";
const char* password = "12345678";

void connectToWiFi() {
  WiFi.begin(ssid, password);
  while (WiFi.status() != WL_CONNECTED) {
    delay(1000);
    Serial.println("Conectando ao Wi-Fi...");
  }
  Serial.println("Conectado ao Wi-Fi");
}
\end{minted}
\clearpage


\subsection{Conexão ao MQTT}

A conexão ao \textit{MQTT Broker} do HAOS foi desenvolvida com base no exemplo
oficial da biblioteca
PubSubClient.\footnote{\url{https://github.com/knolleary/pubsubclient/blob/master/examples/mqtt_basic/mqtt_basic.ino}}

O endereço IPv4 do HAOS, bem como o nome de usuário e a senha de um usuário
secundário, são armazenados em variáveis. A função \texttt{client.setServer()}
é utilizada para definir o endereço IP e a porta do broker MQTT. A seguir, a
função \texttt{client.setCallback()} registra o método que será executado
sempre que uma mensagem for recebida. A conexão é então estabelecida por meio
da função \texttt{client.connect()}, que recebe como parâmetros o identificador
do cliente (no caso, o ESP) e as credenciais. O status da conexão é exibido no
monitor serial, auxiliando na depuração e no monitoramento da comunicação
MQTT.Após a conexão bem-sucedida, o ESP se inscreve nos tópicos de interesse
utilizando a função \texttt{client.subscribe()}.

\begin{minted}{cpp}
const char* mqtt_server = "192.168.88.253";
const char* mqtt_user = "mosquito";
const char* mqtt_password = "12345";

WiFiClient espClient;
PubSubClient client(espClient);

void setupMQTT() {
  client.setServer(mqtt_server, 1883);
  client.setCallback(callback);

  while (!client.connected()) {
    Serial.println("Conectando ao MQTT...");
    if (client.connect("ESP32Client", mqtt_user, mqtt_password)) {
      Serial.println("Conectado ao MQTT");
      client.subscribe("topico/teste");
    } else {
      Serial.print("Falha, rc=");
      Serial.print(client.state());
      delay(5000);
    }
  }
}

void mqttLoop() {
  client.loop();
}
\end{minted}
\clearpage
\subsection{\textit{Callback} MQTT}

A função \texttt{callback}, mencionada anteriormente, tem a função de processar
as mensagens recebidas nos tópicos em que o ESP está inscrito.

O conteúdo da mensagem, inicialmente recebido como um vetor de bytes, é
convertido para uma string, armazenado na variável \texttt{message} e, em
seguida, exibido no monitor serial.

\begin{minted}{cpp}
void callback(char* topic, byte* payload, unsigned int length){
	String message = "";
	for (unsigned int i = 0; i < length; i++) {
		message += (char)payload[i];
	}

	Serial.print("Mensagem recebida [");
	Serial.print(topic);
	Serial.print("]: ");
	Serial.println(message);
}
\end{minted}

A comunicação MQTT foi testada pelo menu do add-on Mosquitto MQTT, publicando
mensagens nos tópicos e verificando a recepção no monitor serial.

\section{Teste dos Sensores}
Com a conexão Wi-Fi e o protocolo de comunicação MQTT funcionando corretamente,
prosseguiu-se com a implementação da leitura dos sensores no microcontrolador,
utilizando como base os exemplos disponibilizados nas bibliotecas dos
respectivos sensores.

\subsection{Sensor BH1750}

O código para leitura do sensor de luminosidade BH1750 foi baseado no exemplo
oficial dabibliotecado sensor.
\footnote{\url{https://github.com/claws/BH1750/blob/master/examples/BH1750test/BH1750test.ino}}

Define-se um tópico MQTT para envio dos dados e uma variável para controle do
intervalo de leitura. A comunicação I2C é inicializada com
\texttt{Wire.begin()}, e o sensor é configurado com
\texttt{lightMeter.begin()}. A cada intervalo, a função
\texttt{lightMeter.readLightLevel()} obtém o valor em lux, que é convertido de
\texttt{float} para \texttt{string} com \texttt{dtostrf()} e publicado via
\texttt{client.publish()} no tópico MQTT correspondente.
\clearpage
\begin{minted}{cpp}
BH1750 lightMeter;
const char* topic_lux = "interruptor/lux";
uint32_t lastLuxReading = 0;

void initLuxSensor() {
  Wire.begin(2, 1);
  lightMeter.begin();
}

void readAndPublishLux() {
  if (millis() - lastLuxReading > 5000) {
    lastLuxReading = millis();
    float lux = lightMeter.readLightLevel();
    char luxStr[8];
    dtostrf(lux, 1, 2, luxStr);
    client.publish(topic_lux, luxStr);
  }
}
\end{minted}

\subsection{Sensor de Radar LD2410}

O código para integração do sensor de radar LD2410 foi desenvolvido com base no
exemplo oficial da biblioteca do sensor.
\footnote{\url{https://github.com/ncmreynolds/ld2410/blob/main/examples/basicSensor/basicSensor.ino}}

Inicialmente, define-se os pinos RX e TX utilizados na comunicação serial com o
radar, além da porta serial secundária (\texttt{Serial1}). A função
\texttt{initRadar()} realiza a inicialização da comunicação serial com o radar e
verifica se a conexão foi bem-sucedida. A função \texttt{readAndPublishRadar()}
é executada periodicamente e realiza a leitura dos dados do sensor. Quando é
detectada uma mudança no estado de presença (presença detectada ou não
detectada), essa informação é publicada em um tópico MQTT específico. Caso haja
presença, também são coletadas e publicadas as informações de distância e
energia dos alvos, tanto estacionários quanto em movimento, nos respectivos
tópicos.
\clearpage
\begin{minted}{cpp}
#define MONITOR_SERIAL Serial
#define RADAR_SERIAL Serial1
#define RADAR_RX_PIN 20
#define RADAR_TX_PIN 21

ld2410 radar;
uint32_t lastReading = 0;
bool presenceState = false;

void initRadar() {
  RADAR_SERIAL.begin(256000, SERIAL_8N1, RADAR_RX_PIN, RADAR_TX_PIN);

  if (radar.begin(RADAR_SERIAL)) {
    MONITOR_SERIAL.println(F("OK"));
  } else {
    MONITOR_SERIAL.println(F("NÃO CONECTADO"));
  }
}

void readAndPublishRadar() {
  if (millis() - lastReading > 1000) {
    lastReading = millis();
    radar.read();

    bool presenceDetected = radar.presenceDetected();

    if (presenceDetected && !presenceState) {
      client.publish("interruptor/radar", "Presença Detectada");
      presenceState = true;
    }
    else if (!presenceDetected && presenceState) {
      client.publish("interruptor/radar", "Presença Não Detectada");
      presenceState = false;
    }

    if (presenceDetected) {
      if (radar.stationaryTargetDetected()) {
        char distStr[8], energyStr[8];
        itoa(radar.stationaryTargetDistance(), distStr, 10);
        itoa(radar.stationaryTargetEnergy(), energyStr, 10);
        client.publish("interruptor/radar/estacionario/distancia", distStr);
        client.publish("interruptor/radar/estacionario/energia", energyStr);
      }

      if (radar.movingTargetDetected()) {
        char distStr[8], energyStr[8];
        itoa(radar.movingTargetDistance(), distStr, 10);
        itoa(radar.movingTargetEnergy(), energyStr, 10);
        client.publish("interruptor/radar/movendo/distancia", distStr);
        client.publish("interruptor/radar/movendo/energia", energyStr);
      }
    }
  }
}
\end{minted}

\section{Juntando todas as funções}
Com a estrutura modular do código pronta e todas as funções definidas nos
cabeçalhos dos arquivos c++ foi feito o código que faz a chamada de todas as
funções usando o Arduino IDE.
\begin{minted}{cpp}
#include "wifi_manager.h"
#include "mqtt_manager.h"
#include "radar_sensor.h"
#include "lux_sensor.h"

void setup() {
  Serial.begin(115200);
  connectToWiFi();
  setupMQTT();
  initLuxSensor();
  initRadar();
}

void loop() {
  mqttLoop();
  readAndPublishLux();
  readAndPublishRadar();
}
\end{minted}

\section{Recebendo os dados no HA}
 Com o \textit{addon} mosquito MQTT instalado no HA, serão adicionadas entidades
 no HA editando o arquivo \texttt{configuration.yaml} as entidades de sensores
 são definidas de acordo com a documentação oficial do HA.
 \footnote{\url{https://www.home-assistant.io/integrations/sensor.mqtt/}}

 \subsection{Definindo os sensores em yaml.}
 Sensores MQTT são definidos em uma tabela yaml que recebe um nome amigável para
 o sensor, um identificador para a entidade do HA e o tópico onde o estado do
 sensor será publicado.

 Será declarado um sensor para cada tópico MQTT definido anteriormente no ESP.

 \begin{minted}{yaml}

mqtt:
  sensor:
    - name: "Lux"
      unique_id: sensor_lux
      state_topic: "interruptor/lux"
      device_class: illuminance
    - name: "Presença Estacionária Distância"
      unique_id: sensor_est_dis
      state_topic: "interruptor/radar/estacionario/distancia"
    - name: "Presença Estacionária Energia"
      unique_id: sensor_est_ene
      state_topic: "interruptor/radar/estacionario/energia"
    - name: "Presença Em Movimento Distância"
      unique_id: sensor_mov_dis
      state_topic: "interruptor/radar/movendo/distancia"
    - name: "Presença Em Movimento Energia"
      unique_id: sensor_mov_ene
      state_topic: "interruptor/radar/movendo/energia"
    - name: "Presença Detectada"
      unique_id: sensor_mov_pre
      state_topic: "interruptor/radar"
\end{minted}

Com os dados armazenados nas entidades elas pode ser usadas de diversas formas
no HA, para os nossos propósitos vamos focar em automações no HA.

\section{Automações}
Automações no HA podem ser definidas no arquivo \texttt{automations.yaml}
ou pelo editor de blocos visual do HA.

\subsection{\textit{Swith} para a lâmpada}

Foi criada uma automação com o objetivo de ligar e desligar a lampada com
base no sensor de presença, para isso é necessário configurar um \textit{switch} MQTT
que é declarado no arquivo \texttt{configuration.yaml} da seguinte forma
\clearpage
\begin{minted}{yaml}
mqtt:
  switch:
    - name: "Lampada"
      unique_id: lampada
      state_topic: "interruptor/switch"
      command_topic: "interruptor/switch"
      payload_on: "1"
      payload_off: "0"
\end{minted}
Com isso foi criada uma entidade \textit{swith} que posta (1) no tópico
\texttt{interruptor/switch} quando o \textit{switch} é ligado e (0) quando
desligado. Devido a natureza do MQTT caso esse estado seja alterado por outro
dispositivo ou de outra forma no HA a entidade também será atualizado de
acordo. Com a entidade definida foram criadas duas automações uma para ligar o
\textit{switch} caso presença seja detectada e desligar o \textit{switch} após
uma quantidade arbitraria de tempo sem presença detectada.

\begin{minted}{yaml}
id: '1743357707824'
  alias: Detector de Presença (Ligar)
  description: Liga a lampada se presença detectada
  triggers:
  - entity_id:
    - sensor.presenca_detectada
    from: Presença Não Detectada
    to: Precença Detectada
    trigger: state
  conditions: []
  actions:
  - action: switch.turn_on
    metadata: {}
    data: {}
    target:
      entity_id: switch.lampada
  mode: single
- id: '1745710625710'
  alias: Sensor de Presença (Desligar)
  description: Desliga a lampada se presença não detectada
  triggers:
  - entity_id:
    - sensor.presenca_detectada
    to: Presença Não Detectada
    trigger: state
    from: Presença Detectada
    for:
      hours: 0
      minutes: 3
      seconds: 0
  conditions: []
  actions:
  - action: switch.turn_off
    metadata: {}
    data: {}
    target:
      entity_id: switch.lampada
  mode: single
\end{minted}
\clearpage

Inscrevendo o ESP nesse tópico torna-se possível acessar essa condição.

Na função \texttt{setupMQTT()}

\begin{minted}{cpp}
void setupMQTT() {
client.subscribe("interruptor/switch");
}
\end{minted}

Para controlar a lâmpada por meio de um relé, pode-se utilizar uma estrutura
condicional \texttt{if}, inserida na função \texttt{callback()}. Nessa função, a
comparação entre a \texttt{string} da variável \texttt{topic} e o tópico
previamente definido no HA é realizada por meio da
função \texttt{strcmp()}. Em seguida, o valor da mensagem contido na variável
\texttt{message} é verificado e, com base nele, a função \texttt{digitalWrite()}
é utilizada para alterar o estado lógico do pino conectado ao relé.

Na função \texttt{callback()}
\begin{minted}{cpp}
const int pinoRELE = 0;
void callback(char* topic, byte* payload, unsigned int length) {
	if (strcmp(topic, "interruptor/switch") == 0) {
		if (message == "1") {
			digitalWrite(pinoRELE, HIGH);
		} else if (message == "0") {
			digitalWrite(pinoRELE, LOW);
		}
	}
}
\end{minted}
\subsection{Controle de brilho da lâmpada}
Em seguida usando os valores lidos e publicados no tópico do sensor lux
para automatizar o controle do brilho da lâmpada.

Adicionada uma entidade \texttt{number} no arquivo \texttt{configuration.yaml} que também é postada em seu
respectivo tópico.
\begin{minted}{yaml}
mqtt:
  number:
    - name: "Brilho"
      unique_id: mqtt_slider_Brilho
      state_topic: "lampada/brilho"
      command_topic: "lampada/brilho"
      min: 0
      max: 100
      step: 1
      retain: false
      unit_of_measurement: "%"
\end{minted}
E criando uma automação usando a formula sugerida nos forums do HA
\footnote{\url{https://community.home-assistant.io/t/dim-lights-as-lumens-increases/182065/15}}
e usando como base o \texttt{blueprint} \footnote{\url{https://community.home-assistant.io/t/smart-lux-dimmer-adjust-light-brightness-depending-on-light-sensor-value/403646}}
\[
	Brilo = (Declive \times  lux) + constante
\]
No arquivo \texttt{automations.yaml} onde teremos as variáveis \texttt{maxB}
Que seria o valor de lux onde o brilho é configurado em 0 como definido na variável \texttt{light\_value\_1},
\texttt{minB} o valor de lux onde o brilho é configurado em 100 como definido na variável \texttt{ligh\_value\_2}.
O declive é calculado com a formula:
\[
slope= \frac{ligh1 - light2}{maxB - minB}
\]
e a constante com a formula:
\[
constant = light1 - (slope \times maxB)
\]


\begin{minted}{yaml}
- id: '1745186830191'
  alias: Dimmer
  description: Configura Brilho baseado em um valor alvo
  triggers:
  - entity_id: sensor.lux
    trigger: state
  conditions:
  - condition: numeric_state
    entity_id: sensor.lux
    above: 0
  actions:
  - target:
      entity_id: number.brilho
    data:
      value: "\n  0\n\n  {{ (( slope
        * states(light_sensor)|int ) + constant)|round }}\n\n"
    action: number.set_value
  variables:
    light_sensor: sensor.lux
    maxB: 400
    minB: 0
    light_value_1: 0
    light_value_2: 100
    light1: '{{ light_value_1 * 1 }}'
    light2: '{{ light_value_2 * 1 }}'
    slope: '{{ ( light1 - light2 ) / ( maxB - minB ) }}'
    constant: '{{ light1 - ( slope * maxB ) }}'
  mode: single
\end{minted}

Novamente o ESP é inscrito nesse tópico.

Na função \texttt{setupMQTT()}
\begin{minted}{cpp}
void setupMQTT() {
	client.subscribe("lampada/brilho");
}
\end{minted}

Na função \texttt{callback()}, são definidas variáveis para o sinal PWM,
incluindo frequência, resolução e o pino correspondente. A função
\texttt{ledcAttach()} do ESP é utilizada para configurar o canal PWM com esses
parâmetros. O valor recebido é restringido entre 0 e 100 e, em seguida,
convertido proporcionalmente para o intervalo de 0 a 255, que representa o
\textit{Duty Cycle} do PWM do ESP. Por fim, a função \texttt{ledcWrite()}
ajusta o pino com o \textit{Duty Cycle} calculado, baseado no valor do brilho
recebido no tópico \texttt{lampada/brilho}.

\begin{minted}{cpp}
const int pwmFreq = 20000;  // Frequência de 20kHz
const int pwmResolution = 8; // Resolução de 8 bits (0-255)
const int pinoPWM = 10;
void setupPWM() {
	ledcAttach(pinoPWM, pwmFreq, pwmResolution);
}
void callback(char* topic, byte* payload, unsigned int length) {
	if (strcmp(topic, "lampada/brilho") == 0) {
		int brilho = constrain(message.toInt(), 0, 100);
		int pwm = (255.0 / 100.0) * brilho;
		ledcWrite(pinoPWM, pwm);
	}
}
\end{minted}

\section{Modificando uma lâmpada}
Lâmpadas de LED comum não podem ter seu brilho ajustado dinamicamente como
lampadas de outras tecnologias podiam.

Para modificar o brilho de lampadas modernas de LED, que usam um simples circuito integrado para controlar a potência da lâmpada foi usado o principio
exemplificado nesse vídeo \footnote{\url{https://www.youtube.com/watch?v=5HTa2jVi_rc}}, mas ao
invéz de modificar o valor dos resistores, o que reduziria a potência da lâmpada
o brilho pode ser controlado dinamicamente usando um transistor que recebe o sinal PWM
do ESP.

\end{document}
